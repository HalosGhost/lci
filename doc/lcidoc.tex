% LCI Documentation (english)
% 
% Copyright (C) 2004-8 Kostas Chatzikokolakis
% This file is part of LCI
%
% This program is free software; you can redistribute it and/or modify
% it under the terms of the GNU General Public License as published by
% the Free Software Foundation; either version 2 of the License, or
% (at your option) any later version.
%
% This program is distributed in the hope that it will be useful,
% but WITHOUT ANY WARRANTY; without even the implied warranty of
% MERCHANTABILITY or FITNESS FOR A PARTICULAR PURPOSE.  See the
% GNU General Public License for more details.

\documentclass[a4paper,11pt]{article}

\addtolength{\hoffset}{-0.4cm}
\addtolength{\textwidth}{0.8cm}

\newcommand{\la}{$\lambda$}
\newcommand{\lci}{\texttt{lci}}
\newcommand{\lc}{$\lambda$-calculus}
\newcommand{\qm}[1]{``#1''}
\newcommand{\kwd}[1]{\texttt{#1}}

\author{
	Kostas Chatzikokolakis
}
\title{
	\kwd{lci} Manual
}
\date{12 March 2006}


\newcommand{\gr}[1]{#1}
\newcommand{\en}[1]{#1}

\newcommand{\tab}{\makebox[.8cm]{}}

\begin{document}

\maketitle
\vspace{60pt}
%\tableofcontents

This manual describes the use of \lci{}, which is an advanced
interpreter for the \lc. This program was first developed by Kostas
Chatzikokolakis as an assignment for the \qm{Theory of Programming Languages} course
in the Department of Informatics of University of Athens. Later it became an
open source project licenced under GPL, in order to be used and improved by
the open source community. It's main purpose is to compute the normal form of
pure \lc{} terms. Moreover it supports various extensions which are, however,
implemented through the pure calculus.

\section{Syntax}
\lci{} supports the syntax of \lc{} enriched with \emph{integers}, \emph{identifiers}
and \emph{operators}. The supported language is described by the following grammar:

\begin{center}
	\begin{tabular}{rcl}
		Term	& $\rightarrow$ & var \\
		& $|$ & ( Term Oper Term ) \\
		& $|$ & ( \la{} var . Term ) \\
		& $|$ & num \\
		& $|$ & id \\
		Oper & $\rightarrow$ & op \\
		& $|$ & $\varepsilon$
	\end{tabular}
\end{center}

The symbol \la{} can be written as \qm{$\backslash$} or as the greek character
\qm{$\lambda$} (using a greek character set). Also ``\verb+->+'' can be used instead
of a dot. \textbf{var} and \textbf{id} are arbitary strings of latin characters
(uppercase or lowercase), numbers and underscores. However \textbf{var} must start
with a lowercase or underscore, while \textbf{id} must start with an
uppercase . It is also possible for any character to be used in an \textbf{id}
if enclosed in single quotes. \textbf{num} is a string of numbers and \textbf{op}
is a string that contains the following characters
\begin{center}
\verb?~ ! @ $ % ^ & * / + - = < > | . , : ; ?
\end{center}
except from the reserved operators \verb+-> . = ; ?+. Finally parentheses can
be avoided according to the following rules
\begin{itemize}
	\item Outmost parentheses can be avoided
	\item Application is left-associative
	\item The scope of an abstraction extends as far to the right as possible
	\item Terms that contain operators are parsed according to the precedence and
	associativity of these operators (see section \ref{sec_oper}).
\end{itemize}

\section{Basic function}
\lci{} is an interactive program. When executed it displays the \verb+lci>+ prompt
and waits for user input. The most simple usage is to enter a \la-term and
press return. The program performs all $\beta$ and $\eta$ reductions and generates
the term's normal form. The result is printed in a \qm{readable} way, that is
only the necessary parentheses are displayed, church numerals are displayed as
integers and lists using the standard Prolog notation. However the way terms
are displayed can be modified, for example the following command
\begin{center}
	\verb+Set showpar on+
\end{center}
causes all parentheses to be displayed.

Terms are reduced using the \emph{normal order evaluation strategy}, that is
the leftmost $\beta$ or $\eta$ reduction is performed first. This strategy
guarantees that term's normal form will be computed in finite time, if it
exists. However if a term has no normal form then execution will not terminate.
After the execution the program displays the number of reductions that were
performed and the CPU usage time.

\section{Integers}
\lci{} supports integers by encoding them as \emph{church numerals} during
parsing. Integer $n$ will be converted to
\[ c_n \equiv \lambda f.\lambda x.f^n(x) \]
So, actually, it is just a syntactic sugar. All operations are implemented in
pure \lc{} and can be used through identifiers and operators. Although the
use of calculus for ordinary operations is sleek, it has has a serious performace
drawback. The complexity of an operation is far from constant, for example
the power-of operator (\verb+**+) requires an exponential number of reductions.
Moreover, due to stack limitaions, the greatest suported integer is 9999.

\section{Identifiers}
Identifiers are used to represent big \la-terms by defining \emph{aliases}.
For example term $\lambda x.x$ can be assigned to alias $I$ so that the term
$I\ y$ is equivalent to $(\lambda x.x)\ y$. Aliases must be defined in a file
that is read by the program. The syntax of this file is described by the following
grammar.
\begin{center}
	\en{
	\begin{tabular}{rcl}
		CmdList & $\rightarrow$ & Cmd ; CmdList \\
		& $|$ & $\varepsilon$ \\
		Cmd & $\rightarrow$ & id = Term \\
		& $|$ & ? Term
	\end{tabular}
	}
\end{center}
Command \kwd{id = Term} assings Term to identifier \kwd{id} while command
\kwd{? Term} causes the evaluation of \kwd{Term} just as if it was entered interactively.
File processing is made using the following command
\begin{center}
	\verb+Consult 'file'+
\end{center}
Morever, when being started, \lci{} searches for a file named \kwd{.lcirc} in the
following places
\begin{verbatim}
	$PREFIX/share/lci/.lcirc  (eg. /usr/local/share/lci/.lcirc)
	$HOME/.lcirc
	./.lcirc
\end{verbatim}
in that order. All files found are executed, if none is found then a warning
is printed. This file contains definitions for many basic
functions and operators (integer operations, for expample) and is similar
to Haskell's \kwd{prelude.hs}.

Identifiers are replaced by the corresponding terms during evaluation and \emph{not}
during parsing. Thus the order of the definitions is not significant. If an alias
is not defined an error message is displayed during evaluation. If no alias
contains itself (directly or indirectly) then aliases are just a syntactic sugar,
for if we replace all of them we get valid \la-terms. However \lci{} supports
curcular references of aliases as a way to implement \emph{recursion}. This
idea is described in the following section.

\section{Recursion}
Recursion is an essential programming tool provided by all serious programming
languages. \lci{} supports two methods of implementing recursion: using
\emph{infinite temrs} or using \emph{fixed point combinators}.

\subsection{Recursion using infinite terms}
\label{par_recinf}
If we allow an alias to contain itself then we can write terms like
\[ M = \lambda x.M\ y \]
Replacing $M$ according to this definition we get the term $\lambda x.(\lambda x.M\ x)\ x$
and if we keep on replacing we get the term
\[ \lambda x.(\lambda x.(\lambda x.(...)\ y)\ y)\ y \]
Thus $M$ can be considered as a term of infinite length. Of course this is not
a valid \la-term, however it can be useful.

Now consider an arbitary closed \la-term $M$, that is $FV(M) = \emptyset$. We
can easily infer the following
\[
	\begin{array}{lcl}
		M[x:=N] & \equiv & M \\
		(\lambda y.N)[x:=M] & \equiv & \lambda y.N[x:=M] \\
		\lambda x.M\ x & \rightarrow_\eta & M
	\end{array}
\]
Thus variable substitution as well as $\eta$-reduction can be performed without
knowing the definition of M, that is without the need to replace it. So during
the evaluation of a term, $M$ must be replaced only if we reach one of the
following terms:
\begin{eqnarray}
	& M\ N \\
	& M
\end{eqnarray}

If the first case $M$ may contain an abstraction and in the second any redex.
If we replace an alias only when necessary, we can finish the evaluation without
performing all the replacements. For example the abstraction $\lambda x.y$ does
not use its argument, so the following reduction
\[ (\lambda x.y)\ M \rightarrow_\beta y \]
eliminates $M$ without computing it.

\lci{} handles identifiers using this techique. That is it replaces an identifier
with it's corresponding term only when necessary and only once at a time. So even
if a term is recursive, it is possible to find a normal form if recursion is
interrupted by some condition. \kwd{.lcirc} contains many recursive definitions,
mainly concerning list manipulation functions.

This technique is not compatible with the pure calculus, as it uses invalid
\la-terms. However the following must be noted: suppose that in term
$M = \lambda x.M\ x$ we need to replace $M$ only twice until we reach it's normal
form. This means that in term $ \lambda x.(\lambda x.(\lambda x.M\ y)\ y)\ y $
no replacement will be performed. $M$. So we can substitute $M$ with an arbitary
valid \la-term $N$ and we get
\[ M' = \lambda x.(\lambda x.(\lambda x.N\ y)\ y)\ y \]
$M'$ behaves exactly like $M$ but it is a valid term. Of course in a different
situation more replacements could be needed, producing a different $M'$. So $M$
could be considered as a \qm{term generator} that produces an appropriate $M'$
each time.

\subsection{Recursion using a fixed point combinator}
\label{par_recfp}
Recursion can be implemented in pure \la-calculus in a very sleek way, using a
fixed point combinator. A such combinator $Y$ is a closed \la-term that
satisfies the following relation
\[ Y f \rightarrow f\ (Y f) \]
for any term $f$. Now let $f$ be a term that satisfies $f=E$, where $E$ is
an expression that containts $f$. We convert $f$ to a variable forming the
non-recursive term $F = \lambda f.E$. It is easy to see that the function we seek
is a fixed point of F, that is $Y F$.

Things get a little bit more complicated if we have the following set of mutually
recursive terms
\begin{eqnarray*}
	f_1 & = & E_1 \\
	& \vdots \\
	f_n & = & E_n
\end{eqnarray*}
where any $f_i$ can be contained in any $E_j$. Now, before applying $Y$, we must
join all terms in one. This can be done using the functions TUPLE $n$ and INDEX $k$.
The former packages $n$ terms into a $n$-tuple, the latter returns the $k$-th
element of a tuple. Both of them can be implemented in pure \lc. So we build
the following recursive term
\[ f = \textrm{\en{TUPLE }} n\ f_i \ \dots\ f_n \]
and replace any occurences of terms $f_i$ using INDEX $k$
\[ f_i \rightarrow \textrm{\en{INDEX}}\ i\ f \]
Finally $f$ is defined using a fixed point combinator
\[ f = Y\ (\lambda f.\textrm{\en{TUPLE }} n\ f_i \ \dots\ f_n ) \]

We have already mentioned that \lci{} allows an alias to contain itself. The
default way of handling such aliases was described in paragraph \ref{par_recinf}.
Moreover \lci{} provides the command \kwd{FixedPoint} which removes circular references
from aliases using a fixed point combinator. Initially this command creates a
graph in which each vertex corresponds to an alias. Two vertexes $S,T$ are connected
with a directed arc if alias $S$ contains $T$ in it's definition. A circle in
this graph denote a set of mutually recursive terms. \kwd{FixedPoint} detects such
circles and, in case they contain more than one arc, it packages the corresponding
terms using functions TUPLE and INDEX. Then it removes recursion using the combinator
$Y$ which must be defined in \kwd{.lcirc}. The modified definition of recursive
aliases can be displayed using the \kwd{ShowAlias} command.

\section{Operators}
\label{sec_oper}
Operators is another tool that is provided by almost all programming languages.
\lci{} supports operators as a special kind of function that takes two arguments
and syntactically appears between them. Using an operator requires two steps.
The first is it's \emph{declaration} together with it's \emph{precedece} and
\emph{associativity}, in a way similar to Prolog. This can be done with the command
\begin{center}
	\kwd{DefOp 'operator' preced assoc}
\end{center}
Quotes are necessary so that operator's name is recognized as an identifier. Precedence
is an integer between 0 and 255 and is used during parsing when no parentheses
are present. Associativity takes one of the following values:
\begin{center}
	\begin{tabular}{cl}
		\kwd{yfx} & Left-associative operator \\
		\kwd{xfy} & Right-associative operator \\
		\kwd{xfx} & Non-associative operator
	\end{tabular}
\end{center}

Character \kwd{x} corresponds to a term with lower precedence than the operator,
while \kwd{y} to one with higher or equal. Thus expression \kwd{a+b+c*d}
will be recognized as \kwd{(a+b)+(c*d)}, for operator \kwd{*} has lower precedence
\footnote{\lci{} behaves similarly to Prolog, that is lower precedence operators
	are applied first.}
than \kwd{+} and \kwd{+} is left-associative. Terms that are not the result of
an operator, or are enclosed in parentheses, are considered to have precedence 0.
Moreover application is considered as a left-associative operator with precedence
100. So if an operator $\kwd{\$}$ is declared with precedence 110 then the
expression \hbox{\kwd{a b\$c}} will be recognized as \hbox{\kwd{(a b)\$c}}.

The second step is operator's \emph{definition} which is performed by defining an
alias with the same name:
\begin{center}
	\kwd{'operator' = ...}
\end{center}
Operator definitions must be placed in a file (as all alias definitions) and quotes
are required. During parsing, \lci{} replaces operators with identifiers, thas is
expression \kwd{a+b} will be transformed to \kwd{'+' a b}. Now \kwd{+} is an
identifier, not an operator, and will be replaced with the corresponding term
during term's evaluation.

In \kwd{.lcirc} many common operators are declared and defined, mainly concerning
integers and list manipulation. These include the right-associative
operator \kwd{:} to write lists as \kwd{a:b:c:Nil}, operator \kwd{++} to append
lists, operator \qm{\kwd{,}} to build ordered pairs \kwd{(a,b)}, integer operations,
integer comparisons etc.

\section{Evaluation strategies}
An evaluation strategy determines the choice of a redex when there are more than
one in a term. \lci{} uses the \emph{normal order} strategy, which selects term's
leftmost redex. The main advantage of this strategy is that it always leads to
term's normal form, if it exists. However it has a serious drawback which is the
multiple computation of terms. For example in the following series of reductions
\begin{eqnarray*}
	&& (\lambda f.f(f\ y))((\lambda x.x)(\lambda x.x)) \\
	& \rightarrow & (\lambda x.x)(\lambda x.x)((\lambda x.x)(\lambda x.x)y) \\
	& \rightarrow & (\lambda x.x)((\lambda x.x)(\lambda x.x)y) \\
	& \rightarrow & (\lambda x.x)(\lambda x.x)y \\
	& \rightarrow & (\lambda x.x)y \\
	& \rightarrow & y
\end{eqnarray*}
the term $(\lambda x.x)(\lambda x.x)$ was computed twice. An alternative strategy
is \emph{call-by-value}, in which all arguments are computed before applied to a function.
This method can avoid multiple computation.
\begin{eqnarray*}
	&& (\lambda f.f(f\ y))((\lambda x.x)(\lambda x.x)) \\
	& \rightarrow & (\lambda f.f(f\ y))(\lambda x.x) \\
	& \rightarrow & (\lambda x.x)((\lambda x.x)y) \\
	& \rightarrow & (\lambda x.x)y \\
	& \rightarrow & y
\end{eqnarray*}
This strategy, however, does not guarantee that normal form will be found. There
are also some other strategies like \emph{call-by-need} that is used in some functional
languages like Haskell.

\lci{} does not implement any such technique, but there has been an effort to
overcome this problem using a special operator $\sim$. This operator does not
behave like ordinary operators. The expression $M\sim N$ denotes the application
of $M$ to $N$ which, however, uses call-by-value. So, if $M\sim N$ is the leftmost
redex then all reductions of $N$ are performed before the application. Thus the
term $(\lambda f.f(f\ y))\sim((\lambda x.x)(\lambda x.x))$ will be reduced according
to the second of the previous ways. Operator $\sim$ has the same precedence and
associativity as the application operator, so it can be easily combined with it.

This operator, however, should be used with caution since the normal form of
$(\lambda x.y)\sim ((\lambda x.x\ x)(\lambda x.x\ x))$ will never be found, yet
it exists. In file \kwd{queens.lci} there is an implementation of the well-known
$n$-queens problem, using experimentally this operator. Without the use of the
operator the program is impossible to terminate, even for 3 queens where the
combinations that must be examined are very few. This is due to the fact that
terms are extremely complex and cause a lot of recomputation. Using the operator
$\sim$ and testing in an Athlon 1800, all solutions for the 3 queens where found in
0.3 seconds, for 4 queens in 4.4 and for 5 in 190. For 6 queens after many hours of
testing the program did not terminate. This is not strange, though, since Haskell
(with the same implementation and using lazy-evaluation and constant time arithmetic)
needs 1799705 reductions for the 8 queens and extremely much time for $n>12$.

\section{\en{Tracing}}
\lci{} supports evaluation tracing. This function is enabled using the following
command
\begin{center}
	\kwd{Set trace on}
\end{center}
or pressing \kwd{Ctrl-C} during the evaluation of a term. When tracing is enabled,
the current term is displayed after each reduction and the program waits for user
input. Available commands are \kwd{step}, \kwd{continue} and \kwd{abort}. The
first one performs the next reduction, the second continues the reductions without
tracing and the last one stops the evaluation. An alternative function is to display
all intermediate terms without interrupting the evaluation. This can be enabled
using the following command
\begin{center}
	\kwd{Set showexec on}
\end{center}

\section{System commands}
In previous paragraphs we have already mentioned some commands that are supported by
\lci{}. These commands are functions that may have arguments. If such a function is
the leftmost in a term, then, instead of evaluating the term, a system command
is executed. All system commands are described in table \ref{tab_syscmd}.

\begin{table}[hbt]
	\centering
	\begin{tabular}{|l|p{200pt}|}
		\hline
		\textbf{Command} & \textbf{Function} \\
		\hline
		\kwd{FixedPoint} & Removes circular references from aliases using a
	  		fixed point combinator $Y$\\
		\kwd{DefOp op prec ass} & Declares an operator with the given precedence and
			associativity. \\
		\kwd{ShowAlias [name]} & Displays the definition of the given alias, or a lists
			of all aliases. \\
		\kwd{Print term} & Displays a term. Useful to check parsing. \\
		\kwd{Consult file} & Reads and processes the given file. \\
		\kwd{Set option on/off} & Changes one of the following parameters. \\
		\multicolumn{2}{|r|}{
			\begin{tabular}{|ll|}
				\hline
				\kwd{trace} & Evaluation tracing \\
				\kwd{showexec} & Display all intermediate terms \\
				\kwd{showpar} & Display all term's parentheses \\
				\kwd{greeklambda} & Display \qm{$\lambda$} instead of \qm{\textbackslash} \\
				\kwd{readable} & Readable display of integers and lists \\
				\hline
			\end{tabular}
		} \\
		\kwd{Help} & Displays a help message. \\
		\kwd{Quit} & Terminates the program. \\
		\hline
	\end{tabular}

	\caption{System commands}
	\label{tab_syscmd}
\end{table}

\section{Examples}
In this section there are some expamples of using the program.

\begin{verbatim}
lci> 3+5*2
13

lci> Sum 1..10
55

lci> Take 10 (Nats 5)
[5, 6, 7, 8, 9, 10, 11, 12, 13, 14]

lci> Map (Add 3) 1..5
[4, 5, 6, 7, 8]

lci> Map (\n.n**2) 1..5
[\y.y, 4, 9, 16, 25]

lci> Filter (Leq 6) 3:6:10:11:Nil
[6, 10, 11]

lci> Length 1..10 ++ 4:5:Nil
12

lci> (Member 3 1..10) && (Length 3:4:5:Nil) >= 3
\x.\y.x
\end{verbatim}

Note that term $\lambda y.y$ is the normal form of number 1. In file \kwd{queens.lci}
there is an implementation of $n$-queens problem.

\begin{verbatim}
lci> Consult 'queens.lci'
Successfully consulted queens.lci
lci> Queens 4
[[2, 4, \y.y, 3], [3, \y.y, 4, 2]]
\end{verbatim}

All of the above functions can be also evaluated using the \kwd{FixedPoint} command,
which removes circular references using the fixed point combinator $Y$. Using
\kwd{ShowAlias} you can see an alias definition after the modification.

\begin{verbatim}
> FixedPoint
> ShowAlias Sum
Sum = Y \_me.\l.If (IsNil l) 0 (+ (Head l) (_me (Tail l)))
\end{verbatim}

\end{document}


